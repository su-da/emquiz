% -*- coding: UTF-8 -*-
\documentclass[a4paper,12pt]{examdesign}
\usepackage{amssymb,amsmath}
\usepackage[range-phrase = \text{\~{}}, range-units = single,
            binary-units]{siunitx}
\usepackage[scale={.8,.85},footskip=40pt]{geometry}
\usepackage{lastpage,float}
\NumberOfVersions{1}
\usepackage[fullfamily,opticals,swash,minionint,openg,lf]{MinionPro}
\usepackage{mathspec}
% \usepackage{unicode-math}
% \usepackage{xeCJK}
\setmainfont{Minion Pro}
\setsansfont{Myriad Pro}
\setmonofont{Consolas}
\sisetup{
    math-micro = \text{μ},
    text-micro = μ,
    math-ohm = \text{Ω},
    text-ohm = Ω,
}
% \setmathfont{XITS Math}
% \setmathfont[range=\mathup/{num,latin,Latin,greek,Greek}]{Minion Pro}
% \setmathfont[range=\mathbfup/{num,latin,Latin,greek,Greek}]{MinionPro-Bold}
% % \setmathfont[range=\mathit/{num,latin,Latin,greek,Greek}]{MinionPro-It}
% \setmathfont[range=\mathbfit/{num,latin,Latin,greek,Greek}]{MinionPro-BoldIt}
% \setmathfont[range=\mathscr,StylisticSet={1}]{XITS Math}
% \setmathfont[range={"005B,"005D,"0028,"0029,"007B,"007D,"2211,"002F,"2215 } ]{Latin Modern Math} % brackets, sum, /
% % \setmathfont[range={"002B,"002D,"003A-"003E} ]{MnSymbol} % + - < = >
% \setmathfont[range={"1D454}]{Latin Modern Math} % openg
% % \setmathrm{Minion Pro}
% \setCJKmainfont[BoldFont={FZSongHei-B07S},
%               ItalicFont={FZKaiTi},
%               SlantedFont={FZFangSongTi}]{FZNewShuSong-Z10}
% \setCJKsansfont[BoldFont={FZDaHei-B02S},
%               ItalicFont={FZLiShu II-S06S},
%               SlantedFont={FZCuYuan-M03S}]{FZHeiTi}
% \setCJKmonofont[BoldFont={FZZhongDengXian-Z07S},
%               ItalicFont={FZXiYuan-M01S},
%               SlantedFont={FZBaoSong-Z04S}]{FZXiDengXian-Z06S}
\defaultfontfeatures{Ligatures=TeX,Scale=MatchLowercase}
\usepackage{graphicx}
\usepackage{tikz}
%\usepackage{floatflt}
\usepackage{array}

\setlength{\parindent}{0pt}
% \setlength{\parskip}{6pt plus 2pt minus 1pt}
% \linespread{1.3}
\usepackage{enumitem}
\setlist{itemsep=0.5\itemsep, parsep=0.5\parsep,
         partopsep=0.5\partopsep, topsep=0.5\topsep}

\def\Varangle#1{\kern.75pt\vtop{\hbox{\kern-.75pt$/{#1}$}\kern-.35pt\hrule}}

\begin{document}
\newcommand{\bhline}{\noalign{\hrule height 1pt}}
\SectionPrefix{}
\NoKey
\NoRearrange
\ShortKey
\ProportionalBlanks{1.5}
\SectionFont{\large\bf}
% \examname{电子技术基础实验}
\DefineAnswerWrapper{\begin{description}\item [\AnswerLeading:]}{\end{description}}
\def\namedata{\large
Number\underline{\hspace{121pt}}Name\underline{\hspace{98pt}}}
% \class{\large 学院(部)\underline{\hspace{98pt}}年级\underline{\hspace{98pt}}专业\underline{\hspace{98pt}}}
\begin{examtop}
\begin{center}
\begin{tabular}{r}
    {\Large \bf QUIZ 1}
    {\large \hspace{48pt}2016 }
\end{tabular} \medskip \\
\namedata
\end{center}
% \bigskip
\end{examtop}

\newcommand\mathdot[1]{\dot#1}

\begin{shortanswer}

\begin{question}
    An air stripline with strips of width $a$ that are a distance $d$
    apart, carries a steady current of intensity $I$. The potential
    difference between the strips is $U$. Prove that the flux of the
    Poynting vector through a cross section of the air stripline is
    $UI$. Neglect the edge effects.
\end{question}

\begin{question}
    The stripline from the preceding problem is connected to a
    sinusoidal generator of emf $U$ and angular frequency $\omega$. The
    other end of the line is connected to an inductor of inductance $L$.
    Calculate the flux of the Poynting vector in complex form through a
    cross section of the air stripline. Check whether the image part of
    the calculated flux is equal to the reactive power of the inductor.
\end{question}

\end{shortanswer}
\end{document}
